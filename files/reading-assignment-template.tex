% CS 8501 Reading Assignment Template
% Author: Yangfeng Ji @ UVa NLP
% Time-stamp: <yj3fs 01/07/2019 12:14:06>

\documentclass[12pt]{article}

\usepackage{graphicx}
\usepackage{fancybox}
\usepackage{epsfig}
\usepackage{float}
\usepackage{amsmath,amsfonts,amssymb}
\usepackage{subfigure}
\usepackage{multirow}
\usepackage{bm}
\usepackage[]{algorithm2e}
\usepackage{color}

\parskip 2.0mm
\topmargin  -2.0cm
\textwidth 16cm
\textheight 22.5cm
\topmargin -1.5cm
\oddsidemargin 0mm

\title{Reading Assignment [1-10]}
\author{Student Name\\ Computing ID}
\date{\today}

% ***************************************************
\begin{document} 
\maketitle
% \tableofcontents

% \begin{abstract}
% \end{abstract}

\begin{itemize}
\item {\it Please keep the format and fill your answer under each question.}
\item {\it Feel free to use mathematical notations in your answer and make sure to provide explanations for them.}
\item {\it For each question (except the first one), a short paragraph with 2 - 4 sentences should be enough.}
\end{itemize}

\section*{Questions}

\begin{enumerate}
\item Paper title
\item What is the {\bf specific} research problem addressed in this paper?\\
  Note: {\it Make sure you actually identify the research problem. For example, answers like ``improve the text classification performance'' or ``further reduce the perplexity in language modeling'' are not research problems --- they are the outcomes of solving the problems.}
\item What is the {\bf proposed method}? How it can address the problem?
\item What are the {\bf main challenges} of the proposed method?\\
  Note: {\it There could be multiple challenges of solving this problem, for example, on modeling, training and data collection.}
\item Based on the related work section of this paper, what are the {\bf closely} related methods in prior work? How they addressed the same or similar problem?
\item What are the main observations from the experiments (if any)?
\end{enumerate}

\end{document}

